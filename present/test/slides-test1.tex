\documentclass[14pt,xcolor=dvipsnames]{beamer}
 
% Specify theme
\usetheme{progressbar}
% See http://goo.gl/Wxlyy for alternative themes

% Specify font
% \usepackage{iwona}
 
% Specify base color
\usecolortheme[named=OliveGreen]{structure}
% See http://goo.gl/p0Phn for alternative colors
 
% Specify other colors and options as required
\setbeamercolor{alerted text}{fg=Maroon}
\setbeamertemplate{items}[square]
 
% Title and author information
\title{Title goes here}
\author{Name goes here}
 
 
\begin{document}
 
\begin{frame}
\titlepage
\end{frame}
 
\begin{frame}{Outline}
\tableofcontents
\end{frame}
 
\section{Introduction}
 
\begin{frame}{Forecasting functional data}
\begin{itemize}[<+-| alert@+>]
\item Observed values are discrete but underlying structures are 
continuous functions.
\item Observed values may be noisy but underlying functions are 
smooth.
\item \textbf{Problem:} To forecast the \textbf{whole function} for 
future time periods.
\end{itemize}
\end{frame}
 
\begin{frame}{Forecasting functional data}
\structure{Some notation}
 
Let $y_t(x_i)$ be the observed data in period $t$ at location $x_i$, 
$i=1,\dots,p$, $t=1,\dots,n$.
\pause
 
\begin{block}{}
\[
y_t(x_i) = f_t(x_i) + \sigma_t(x_i)\varepsilon_{t,i}
\]
where $\varepsilon_{t,i}$ is iid N(0,1) and $\sigma_t(x_i)$ allows the 
amount of noise to vary with $x$.
\end{block}
\pause
 
\begin{enumerate}[<+-| alert@+>]
\item We assume $f_t(x)$ is a smooth function of $x$.
\item We need to estimate $f_t(x)$ from the data for $x_1 < x < x_p$.
\end{enumerate}
\end{frame}
 
\end{document}
