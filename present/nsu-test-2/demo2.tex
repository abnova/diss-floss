\documentclass[9pt,english]{beamer}  
 
\usepackage[T1]{fontenc}
\usepackage[utf8x]{inputenc}
\usepackage[english]{babel}
\usepackage{lmodern}
\usepackage{colortbl}
\usepackage{bibunits}
\usepackage{marvosym}

\usepackage{stmaryrd}                  % Pour avoir les d\'elimiteurs entiers
\usepackage{booktabs}
\usepackage{tabularx}
\usepackage{xspace}

\newcommand{\Omit}[1]{}
\newcommand{\fixme}[1]{{\small\color{red}[FIXME : #1]}}
\definecolor{mybeautifulcyan}{rgb}{0.08,0.58,0.89}
\newcommand{\sylvain}[1]{{\small\color{mybeautifulcyan}[Sylvain : #1]}}
\newcommand{\michel}[1]{{\small\color{green}[Michel : #1]}}

% --------- ensembles célèbres
\newcommand{\N}{\mathbb{N}}
\newcommand{\R}{\mathbb{R}}
\newcommand{\Rplus}{\R^+}


% -------- variables
%\newcommand{\Var}[1]{\mathbf{#1}}
%\newcommand{\Dom}[1]{\mathrm{Dom}(#1)}
%\newcommand{\EnsVar}[1]{\mathbf{#1}}

\newcommand{\ens}[1]{\mathcal{#1}}
%\newcommand{\vect}[1]{\boldsymbol{\mathbf{#1}}}        
\newcommand{\vect}[1]{\overrightarrow{#1}}
\newcommand{\setsof}[1]{\wp(#1)}     
\newcommand{\mat}[1]{\vect{#1}}
%\newcommand{\matrices}[3]{\mathcal{M}_{#1, #2}(#3)}
\newcommand{\multiens}[1]{\{\!\!\{#1\}\!\!\}}

%--------- logique
% \newcommand{\If}{\mathbf{if}}
% \newcommand{\Otherwise}{\mathbf{otherwise}}

%--------- divers
\newcommand{\Def}{\buildrel\hbox{\tiny \textit{def}}\over =}
\newcommand{\tq}{\ |\ }
\DeclareMathOperator*{\esp}{E}
\DeclareMathOperator*{\esppr}{E_{\Pr}}
\DeclareMathOperator{\argmax}{argmax}
\DeclareMathOperator{\argmin}{argmin}
\DeclareMathOperator{\leximin}{leximin}
\DeclareMathOperator{\sign}{sign}
\DeclareMathOperator{\variance}{Var}
\newcommand{\Variance}[1]{\variance(#1)}

\newtheorem{defin}{Définition}
\newtheorem{prop}{Proposition}
\newtheorem{exemple}{Exemple}
\newtheorem{remarque}{Remarque}
\newtheorem{lemme}{Lemme}
\newtheorem{conjecture}{Conjecture}
\newtheorem{corollaire}{Corollaire}

%\renewcommand{\qedsymbol}{$\blacksquare$}
\theoremstyle{nonumberplain}
%\theorembodyfont{\normalfont}
%\theoremsymbol{\qedsymbol}

%------------ macros specifiques pour l'article

% lettre pour agents, objets
\newcommand{\Ag}{A} 
\newcommand{\Ob}{O}
\newcommand{\ag}{i}
\newcommand{\agp}{j}
\newcommand{\ob}{o}

\newcommand{\Agents}{{\ens{\Ag}}}
\newcommand{\maxAgents}{{n}}
\newcommand{\Objets}{{\ens{\Ob}}}
\newcommand{\maxObjets}{{m}}

\newcommand{\score}{g}
\newcommand{\rang}{r}

\newcommand{\Poids}{W}
\newcommand{\poids}{w}

\newcommand{\share}{\pi}
\newcommand{\Shares}{\vect{\share}}

\newcommand{\price}{p}
\newcommand{\Prices}{\vect{\price}}

\newcommand{\Ut}{u}     % utilité individuelle
\newcommand{\Utc}{cu}   % utilité collective

\newcommand{\pb}{problème de partage de biens indivisibles à
préférences additives\xspace}
\newcommand{\jpp}{juste part proportionnelle\xspace}
\newcommand{\jpM}{juste part max-min\xspace}
\newcommand{\jpm}{juste part min-max\xspace}


\newcommand{\PFS}{\mathrm{PFS}}   % proportional fair share property
\newcommand{\MFS}{\mathrm{MFS}} % maxmin fair share property
\newcommand{\mFS}{\mathrm{mFS}}  % minmax fair share property
\newcommand{\EF}{\mathrm{EF}}  % envy-free property
\newcommand{\CEEI}{\mathrm{CEEI}}  % CEEI property

\newcommand{\Instances}{\mathcal{I}}
\newcommand{\Inst}[1]{\mathcal{I}_{| #1}}
\newcommand{\InstPFS}{\Inst{\PFS}}   % proportional fair share property
\newcommand{\InstMFS}{\Inst{\MFS}} % maxmin fair share property
\newcommand{\InstmFS}{\Inst{\mFS}}  % minmax fair share property
\newcommand{\InstEF}{\Inst{\EF}}  % envy-free property
\newcommand{\InstCEEI}{\Inst{\CEEI}}  % CEEI property

\newcommand{\Verif}[2]{#2 \vDash #1}

\newcommand{\uPFS}{u^\PFS}   
\newcommand{\uMFS}{u^\MFS}
\newcommand{\umFS}{u^\mFS}

\newcommand{\Adm}{{\cal F}}

\newcommand{\llb}{\llbracket}
\newcommand{\rrb}{\rrbracket}

\newcommand{\pol}{\textsf{P}\xspace}
\newcommand{\np}{\textsf{NP}\xspace}
\newcommand{\conp}{\textsf{coNP}\xspace}
\newcommand{\DP}{\textsf{DP}\xspace}
\newcommand{\bhdeux}{$\mathsf{BH_2}$\xspace}
\newcommand{\Prob}[1]{\textsc{[#1]}}
\newcommand{\sigmaDeuxP}{$\mathsf{\Sigma_2^P}$\xspace}

\newcolumntype{L}{>{\raggedright\arraybackslash}X}
\newcolumntype{R}{>{\raggedleft\arraybackslash}X}
\newcolumntype{C}{>{\centering\arraybackslash}X}

\newcounter{problem}
\newcommand{\EP}[3]{
  \refstepcounter{problem}
  \smallskip
  \begin{center}
    {\small
      \begin{tabularx}{\columnwidth}{ll}
        \toprule
        \multicolumn{2}{c}{Problème~\theproblem\quad\Prob{#1}} \\
        \midrule
        {\bf Entrée :} & \parbox[t]{0.70\columnwidth}{#2\vspace*{1mm}} \\
        {\bf Question :}& \parbox[t]{0.70\columnwidth}{#3\vspace*{.5mm}} \\
        \bottomrule
      \end{tabularx}
    }
  \end{center}
  \smallskip
}



\makeatletter
\def\pgfsincos#1{%
  \pgfmathparse{#1}%
  \pgfmathcos@{#1}\pgf@y=\pgfmathresult pt%
  \pgfmathsin@{#1}\pgf@x=\pgfmathresult pt%
}
\makeatother

\newcommand{\fup}[1]{\raisebox{0.5em}{\scriptsize #1}}

%\usetheme{steamer}
\usetheme{progressbar}
\progressbaroptions{headline=none,frametitle=picture-subsection,titlepage=picture,imagename=images/LogoNSU3.jpg}
\setbeamerfont{title}{size=\small,series=\bfseries,parent=structure}
\setbeamerfont{author}{size=\small,series=\bfseries,parent=structure}
\setbeamerfont{date}{size=\small,series=\bfseries,parent=structure}

\usetikzlibrary{shapes.misc,shapes.geometric,shapes.callouts}

%\setbeamercolor{custombox}{bg=steamer@blue!20!structure.bg,fg=structure.fg}

\usepackage{mytitlepage}

\title{\centering{Governance and Organizational Sponsorship\\
                  as Success Factors in\\
                  Free/Libre and Open Source Software Development:\\
                  An Empirical Investigation\\
                  using Structural Equation Modeling\\
                  \parbox{0.7\textwidth}}}

% \parbox{0.7\textwidth}
\author{\centering{
  Aleksandr Blekh\\
  Nova Southeastern University\\
  Graduate School of Computer and Information Sciences\\
  \parbox{0.7\textwidth}}}

\date{\centering{
  Defence of a dissertation submitted in partial fulfillment of the requirements\\
  for the degree of Doctor of Philosophy in Information Systems\\
  Fort Lauderdale, FL, USA\\
  November 20, 2014\\}}


\begin{document}

\begin{frame}[plain]
  \titlepage
\end{frame}

\section{The problem}

\begin{frame}{Fair division of indivisible goods}
  \begin{overlayarea}{\textwidth}{0.5\textheight}
    We have:
    \begin{itemize}
    \item<2-> a finite set of \emph{objects} $\Objets = \{1,\dots,\maxObjets\}$
    \item<3-> a finite set of \emph{agents} $\Agents =
      \{1,\dots,\maxAgents\}$
    \item<4-> \emph{Additive preferences}: \parbox[t]{0.5\textwidth}{$\to$ $w_{\ag}(\ob)$ is the
      weight of object $\ob$ for agent $\ag$\only<5->{\\$\to$
        $u_{\ag}(\mathcal{X}) = \sum_{\ob \in \mathcal{X}}
        w_{\ag}(\ob)$}.}
    \end{itemize}
    
    \only<6->{We want:}
    \begin{itemize}
    \item<7-> a \emph{complete allocation} $\Shares: \Agents \to
      2^{\Objets}$...
    \item<8-> ...which takes into account the agents' preferences
      (\emph{fair allocation}).
    \end{itemize}

    \only<9>{In this work, we consider five \emph{fairness} criteria.}
  \end{overlayarea}
  
  \vfill

  \begin{overlayarea}{\textwidth}{0.2\textheight}
    \only<2>{
      \parbox{0.45\textwidth}{        
        \begin{tabular}{cccc}
          \phantom{agent 007} & \tikz\node[fill=structure.fg!20!bg,circle]{1}; & \tikz\node[fill=structure.fg!20!bg,circle]{2}; & \tikz\node[fill=structure.fg!20!bg,circle]{3};\\
        \end{tabular}      
      }
    }
    \only<3>{
      \parbox{0.45\textwidth}{
        \begin{tabular}{cccc}
          \phantom{agent 007} & \tikz\node[fill=structure.fg!20!bg,circle]{1}; & \tikz\node[fill=structure.fg!20!bg,circle]{2}; & \tikz\node[fill=structure.fg!20!bg,circle]{3};\\
          agent $1$ & & & \\
          agent $2$ & & & \\
        \end{tabular}      
      }
    }
    \only<4->{
      \parbox{0.45\textwidth}{
        \begin{tabular}{c|c|c|c}
          \phantom{agent 007} & \tikz\node[fill=structure.fg!20!bg,circle]{1}; & \tikz\node[fill=structure.fg!20!bg,circle]{2}; & \tikz\node[fill=structure.fg!20!bg,circle]{3};\\\hline
          agent $1$ & {\only<7,8,9>{\cellcolor{structure.fg!90!bg}\color{bg}}5} & 4 & {2}\\\hline
          agent $2$ & {4} & {\only<5,7,8,9>{\cellcolor{structure.fg!90!bg}\color{bg}}1} & {\only<5,7,8,9>{\cellcolor{structure.fg!90!bg}\color{bg}}6}\\
        \end{tabular}      
      }
    }
    \only<5>{
      \hfill\parbox{0.45\textwidth}{
        $u_2(\{2, 3\}) = 1 + 6 = 7$
      }
    }
    \only<7->{
      \hfill\parbox{0.45\textwidth}{
        $\Shares = \langle \{1\},\{2, 3\} \rangle$
        \begin{itemize}
        \item $u_1(\Shares) = 5$
        \item $u_2(\Shares) = 7$
        \end{itemize}
      }
    }
  \end{overlayarea}
\end{frame}


% \begin{frame}{Fair division of indivisible goods}
%   Fair division of indivisible goods\dots

%   \vfill

%   We have:
%   \begin{itemize}
%   \item a finite set of \emph{objects} $\Objets = \{1,\dots,\maxObjets\}$
%   \item a finite set of \emph{agents} $\Agents = \{1,\dots,\maxAgents\}$
%     having some \emph{preferences} on the set of objects they may
%     receive
%   \end{itemize}

%   \vfill\pause

%   We want:
%   \begin{itemize}
%   \item an allocation $\Shares:\Agents \to 2^{\Objets}$
%   \item such that $\share_i \cap \share_j = \emptyset$ if $i \neq j$
%     (preemption),
%   \item $\bigcup_{\ag \in \Agents} \share_i = \Objets$ (no
%     free-disposal),
%   \item and which takes into account the agents' preferences
%   \end{itemize}

%   \vfill\pause

%   Plenty of real-world applications: course allocation, operation of
%   Earth observing satellites, \dots
% \end{frame}

% \begin{frame}{Centralized allocation}
%   \begin{bibunit}[apalike]
%     A classical way to solve the problem:
    
%     \begin{itemize}
%     \item Ask each agent $\ag$ to give a score (weight, utility\dots) $w_\ag(\ob)$
%       to each object $\ob$
%     \item Consider all the agents have \emph{additive} preferences\\
%       \ \hfill $\to$ $u_\ag(\share) = \sum_{\ob \in \share} w_\ag(\ob)$
%     \item Find an allocation $\Shares$ that satisfies a given \emph{fairness criterion},\\
%       \ \hfill \textit{e.g.} $u_\ag(\share_\ag) \geq
%       u_\ag(\share_\agp)$ for all agents $\ag, \agp$ -- envy-freeness\\
%       \ \hfill \cite{Lipton04}.
%     \end{itemize}

%     \begin{beamercolorbox}[wd=\textwidth,sep=1ex]{custombox}%
%       \putbib[demo2bib]
%     \end{beamercolorbox}
%   \end{bibunit}
% \end{frame}

% \begin{frame}{Example}
%   \textbf{Example:} 3 objects $\{1, 2, 3\}$, 2 agents $\{1, 2\}$.

%   \vfill\pause
  
%   \emph{Preferences:}

%   \ 

%   \begin{tabular}{|c|c|c|c|}
%     \hline
%     & $1$ & $2$ & $3$\\\hline
%     agent $1$ & {\only<3->{\cellcolor{structure.fg!90!bg}\color{bg}}5} & {\only<4>{\cellcolor{structure.fg!90!bg}\color{bg}}4} & {2}\\\hline
%     agent $2$ & {4} & {\only<3>{\cellcolor{structure.fg!90!bg}\color{bg}}1} & {\only<3->{\cellcolor{structure.fg!90!bg}\color{bg}}6}\\\hline
%   \end{tabular}
  
%   \vfill
  
%   \begin{overlayarea}{\textwidth}{2cm}
%     \only<3->{\emph{Envy-freeness:}}
    
%     \only<3->{$\Shares = \langle \{1\},\{2, 3\} \rangle$ is \emph{not} envy-free (agent $1$ envies agent $2$)\\}
%     \only<4->{$\Shares' = \langle \{1, 2\},\{3\} \rangle$ is envy-free.}
%   \end{overlayarea}
% \end{frame}

% \begin{frame}{Fairness properties}
%   Choose a \emph{fairness property}, and find an allocation that
%   satisfies it\dots

%   \pause\vfill

%   \textbf{Problems:}
%   \begin{enumerate}
%   \item such an allocation does not always exist\\
%     \ \hfill $\to$ \textit{e.g.} 2 agents, 1 object: no envy-free
%     allocation exists
%   \item many such allocations can exist
%   \end{enumerate}
  
%   \pause\vfill

%   \textbf{Idea:} consider several fairness properties, and try to
%   satisfy the strongest one.

%   In this work we consider five such properties.
% \end{frame}

\section{Five fairness criteria}

% \begin{frame}{Outline}
%   \tableofcontents[current]
% \end{frame}


\begin{frame}{Envy-freeness}
  \begin{bibunit}[apalike]
    \begin{block}{Envy-freeness}
      An allocation $\Shares$ is \emph{envy-free} if no agent envies
      another one.
    \end{block}
    
    \pause\vfill

    \textbf{Example:} 3 objects $\{1, 2, 3\}$, 2 agents $\{1, 2\}$.
    
    \vfill\pause
    
    \emph{Preferences:}
    
    \ 
    
    \begin{tabular}{|c|c|c|c|}
      \hline
      & $1$ & $2$ & $3$\\\hline
      agent $1$ & {\only<4->{\cellcolor{structure.fg!90!bg}\color{bg}}5} & {\only<5>{\cellcolor{structure.fg!90!bg}\color{bg}}4} & {2}\\\hline
      agent $2$ & {4} & {\only<4>{\cellcolor{structure.fg!90!bg}\color{bg}}1} & {\only<4->{\cellcolor{structure.fg!90!bg}\color{bg}}6}\\\hline
    \end{tabular}
    
    \vfill
    
    \begin{overlayarea}{\textwidth}{2cm}
      \only<4->{\emph{Envy-freeness:}}
      
      \only<4->{$\Shares = \langle \{1\},\{2, 3\} \rangle$ is \emph{not} envy-free (agent $1$ envies agent $2$)\\}
      \only<5->{$\Shares' = \langle \{1, 2\},\{3\} \rangle$ is envy-free.}
    \end{overlayarea}
  \end{bibunit}
\end{frame}

\begin{frame}{Envy-freeness}
  \begin{bibunit}[apalike]
    \begin{block}{Envy-freeness}
      An allocation $\Shares$ is \emph{envy-free} if no agent envies
      another one.
    \end{block}
    
    \textbf{Known facts:}
    \begin{itemize}
    \item An envy-free allocation may not exist.
    \item Deciding whether an allocation is envy-free is easy
      (quadratic time).
    \item Deciding whether an instance (agents, objects, preferences)
      has an envy-free allocation is hard -- \textbf{NP}-complete
      \cite{Lipton04}.
    \end{itemize}

    \vfill

    \begin{beamercolorbox}[wd=\textwidth,sep=1ex]{custombox}%
      \putbib[demo2bib]
    \end{beamercolorbox}
    
    % \pause\vfill

    % \begin{center}
    %   \begin{tikzpicture}\usebeamercolor{structure}\small
    %     \draw[thick, fg, ->, >=latex] (0, 0) -- (0.8\textwidth, 0);
    %     \draw[fg] (0, 0) node[anchor=north] {\footnotesize weaker};
    %     \draw[fg] (0.8\textwidth, 0) node[anchor=north] {\footnotesize stronger};
        
    %     \draw[fg, thick] (0.55\textwidth, -0.5ex) -- (0.55\textwidth, 0.5ex) node[above, draw=fg, fill=fg!30!bg, rounded corners] {\bf envy-freeness};
    %   \end{tikzpicture}
    % \end{center}
  \end{bibunit}
\end{frame}

\begin{frame}{Proportional fair share}
  \begin{bibunit}[apalike]
    \textbf{Proportional fair share (PFS):}
    \begin{itemize}
    \item Initially defined by \cite{Steinhaus48} for continuous fair
      division (\textit{cake-cutting})
    \item \textbf{Idea:} each agent is ``entitled'' to at least the
      n\fup{th} of the entire resource
    \end{itemize}
    \vfill

    \begin{beamercolorbox}[wd=\textwidth,sep=1ex]{custombox}%
      \putbib[demo2bib]
    \end{beamercolorbox}
    
    \pause\vfill
  
    \begin{block}{Proportional fair share}
      The \emph{proportional fair share} of an agent $\ag$ is equal
      to:
      \[
      \uPFS_\ag \Def \frac{u_i(\Objets)}{\maxAgents} = \sum_{\ob \in \Objets} \frac{w_i(\ob)}{\maxAgents}
      \]

      An allocation $\Shares$ satisfies \emph{(proportional) fair
        share} if every agent gets at least her fair share.
    \end{block}
  \end{bibunit}
\end{frame}  

\begin{frame}{Proportional fair share: facts}
  \textbf{Easy or known facts:}
  \begin{itemize}
  \item Deciding whether an allocation satisfies proportional fair
    share (PFS) is easy (linear time).
  \item For a given instance, there may be no allocation satisfying
    PFS\\
    \ \hfill $\to$ \textit{e.g.} 2 agents, 1 object
  \item This is not true for cake-cutting (divisible resource)\\
    \ \hfill$\to$ Dubins-Spanier
  \end{itemize}

  \pause\vfill

  \textbf{New (?) facts:}
  \begin{itemize}
  \item Deciding whether an instance has an allocation satisfying PFS
    is hard even for 2 agents -- \textbf{NP}-complete
    [\textsc{Partition}].
  \item $\Shares$ is envy-free $\Rightarrow$ $\Shares$ satisfies
    PFS$^{\text{1}}$.
  \end{itemize}
  {\scriptsize $^{\text{1}}$ Actually already noticed at least in an
    unpublished paper by Endriss, Maudet \textit{et al.}}

  \pause\vfill
  
  \begin{center}
    \begin{tikzpicture}\usebeamercolor{structure}\small
      \draw[thick, fg, ->, >=latex] (0, 0) -- (0.8\textwidth, 0);
      \draw[fg] (0, 0) node[anchor=north] {\footnotesize weaker};
      \draw[fg] (0.8\textwidth, 0) node[anchor=north] {\footnotesize stronger};
      
      \draw[fg, thick] (0.55\textwidth, -0.5ex) -- (0.55\textwidth, 0.5ex) node[above] {\bf envy-freeness};
      \draw[fg, thick] (0.25\textwidth, -0.5ex) -- (0.25\textwidth, 0.5ex) node[above] {\bf proportional fair share};
    \end{tikzpicture}
  \end{center}
\end{frame}

\begin{frame}{Max-min fair share}
  PFS is nice, but sometimes too demanding for indivisible goods\\
  \ \hfill $\to$ \textit{e.g.} 2 agents, 1 object

  \pause\vfill

  \begin{bibunit}[apalike]
    \textbf{Max-min fair share (MFS):}
    \begin{itemize}
    \item Introduced recently \cite{Budish11}; not so much studied so
      far.
    \item \textbf{Idea:} in the \emph{cake-cutting} case, PFS = the
      best share an agent can hopefully get for sure in a \textit{``I
        cut, you choose (I choose last)''} game.
    \item Same game for indivisible goods $\to$ MFS.
    \end{itemize}
    
    \vfill
    
    \begin{beamercolorbox}[wd=\textwidth,sep=1ex]{custombox}%
      \putbib[demo2bib]
    \end{beamercolorbox}    
  \end{bibunit}
\end{frame}

\begin{frame}{Max-min fair share}
  \textbf{Idea:} in the \emph{cake-cutting} case, PFS = the
  best share an agent can hopefully get for sure in a \textit{``I
    cut, you choose (I choose last)''} game.

  \begin{block}{Max-min fair share}
    The \emph{max-min fair share} of an agent $\ag$ is equal to:
    \[
    \uMFS_\ag  \Def \max_{\Shares} \min_{\agp \in \Agents } u_\ag(\share_{\agp})
    \]
    
    An allocation $\Shares$ satisfies \emph{max-min fair share}
    (MFS) if every agent gets at least her max-min fair share.
  \end{block}
\end{frame}

\begin{frame}{Max-min fair share: examples}
  \textbf{Example:} 3 objects $\{1, 2, 3\}$, 2 agents $\{1, 2\}$.

  \vfill
  
  \emph{Preferences:}

  \ 

  \begin{tabular}{|c|c|c|c|c}
    \cline{1-4}
    & $1$ & $2$ & $3$ & \\\cline{1-4}
    agent $1$ & {\only<4>{\color{fg!30!bg}}5} & {\only<3>{\color{fg!30!bg}}4} & {\only<3>{\color{fg!30!bg}}2} & \only<2->{$\to \uMFS_1 = 5$ (with cut $\langle \{1\}, \{2, 3\}\rangle$)}\\\cline{1-4}
    agent $2$ & {\only<3>{\color{fg!30!bg}}4} & {\only<4>{\color{fg!30!bg}}1} & {\only<4>{\color{fg!30!bg}}6} & \only<2->{$\to \uMFS_2 = 5$ (with cut $\langle \{1, 2\}, \{3\}\rangle$)} \\\cline{1-4}
  \end{tabular}

  \vfill

  \begin{overlayarea}{\textwidth}{2cm}
    \only<3->{\emph{MFS evaluation:}}
  
    \only<3->{$\Shares =  \langle \{1\},\{2, 3\} \rangle \to u_1(\share_1) = 5 \geq 5$; $u_2(\share_2) = 7 \geq 5$ $\Rightarrow$ MFS satisfied\\}
    \only<4->{$\Shares'' =  \langle \{2, 3\},\{1\} \rangle \to u_1(\share''_1) = 6 \geq 5$; $u_2(\share''_2) = 4 {\ \color{red} < 5}$ $\Rightarrow$ MFS not satisfied}
  \end{overlayarea}

  \vfill

  \only<5->{\hrule}

  \vfill

  \begin{overlayarea}{\textwidth}{2cm}
    \only<5->{
      \textbf{Example:} 2 agents, 1 object.
    }
    \only<6>{\\
      \ \hfill $\uMFS_1 = \uMFS_2 = 0$ $\to$ every allocation satisfies
      MFS!\\
      Not very satisfactory, but can we do much better?
    }
  \end{overlayarea}
\end{frame}

\begin{frame}{Max-min fair share: properties}
  \textbf{Facts:}
  \begin{itemize}
  \item Computing $\uMFS_\ag$ for a given agent is hard $\to$
    \textbf{NP}-complete [\textsc{Partition}]
  \item Hence, deciding whether an allocation satisfies MFS is also
    hard.
  \item $\Shares$ satisfies PFS $\Rightarrow$ $\Shares$ satisfies MFS.
  \end{itemize}

  \pause\vfill
  
  \begin{center}
    \begin{tikzpicture}\usebeamercolor{structure}\small
      \draw[thick, fg, ->, >=latex] (0, 0) -- (0.8\textwidth, 0);
      \draw[fg] (0, 0) node[anchor=north] {\footnotesize weaker};
      \draw[fg] (0.8\textwidth, 0) node[anchor=north] {\footnotesize stronger};
      
      \draw[fg, thick] (0.55\textwidth, -0.5ex) -- (0.55\textwidth, 0.5ex) node[above] {\bf envy-freeness};
      \draw[fg, thick] (0.25\textwidth, -0.5ex) -- (0.25\textwidth, 0.5ex) node[above] {\bf proportional fair share};
      \draw[fg, thick] (0.1\textwidth, -0.5ex) -- (0.1\textwidth, 0.5ex) node (MFS) {};
      \draw[fg, <-, >=latex] (MFS) -- ++(0, 1.5em) node[above, draw=fg, fill=fg!30!bg, rounded corners] {\bf max-min fair share};
    \end{tikzpicture}
  \end{center}

  % \textbf{Intuition:}
  % \begin{itemize}
  % \item the situation where all agents have the same preferences is
  %   the \emph{worst} possible situation
  % \item in that situation, an allocation satisfying MFS exists (see
  %   definition)
  % \item all other situation makes every agent better off.
  % \end{itemize}
\end{frame}


\begin{frame}{Max-min fair share: conjecture}
  \begin{alertblock}{Conjecture}
    \tikz[remember picture]\node (conj) {};\!\!\!For each instance there is at least one allocation satisfying
    max-min fair share.
  \end{alertblock}

  \pause\vfill

  \begin{itemize}
  \item Proved for \textbf{special cases} (2 agents, matching,\dots),
    even \emph{very general ones} (scoring functions\dots)
  \item No counterexample found on \emph{thousands} of \textbf{random}
    instances.
  \end{itemize}

  \pause\vfill

  \begin{bibunit}[apalike]
    \begin{tikzpicture}[remember picture,overlay]
      \draw (conj) ++ (0.45\textwidth, 0) node[rounded corners, rotate=15, opacity=0.8, fill=black!50!red!50!white, draw=black!50!red, very thick, minimum width=3.8cm, minimum height=1.5cm] {\Huge\color{black!50!red}\textbf{FALSE!}};
    \end{tikzpicture}
    
    The conjecture has been proved \emph{false} by Procaccia and
    Wang\nocite{Procaccia14} using a \emph{very tricky} counterexample.

    \vfill
    
    \begin{beamercolorbox}[wd=\textwidth,sep=1ex]{custombox}%
      \putbib[demo2bib]
    \end{beamercolorbox}    
  \end{bibunit}
\end{frame}



% \begin{frame}{Max-min fair share: special cases}
%   \textbf{Special cases:} conjecture proved for:
%   \begin{itemize}[<+->]
%   \item Agents having same preferences (see definition)
%   \item 2 agents: ``I cut, you choose''
%   \item $\maxObjets < \maxAgents$ (strictly less objects than agents)
%     or $\maxObjets = \maxAgents$ (matching)
%   \item Preferences represented by \emph{scoring functions}:
%     \begin{itemize}
%     \item Each agent $\ag$ ranks all the objects (\textit{e.g} $3 \succ_\ag
%       1 \succ_\ag 2 \succ_\ag 4$)
%     \item A common \emph{scoring function} maps ranks to scores\\
%       \ \hfill $g: \{1, \dots, \maxObjets\} \to \mathbb{N}$
%     \item The weight of object $\ob$ for agent $\ag$ is computed using
%       this function:\\
%       \ \hfill $w_\ag(\ob) = g(rank_\ag(\ob))$.
%     \end{itemize}
%   \end{itemize}

%   \pause\vfill

%   \textbf{Experiments:} no counterexample found on thousands of random
%   instances.

%   \pause\vfill
  
%   \begin{center}
%     \begin{tikzpicture}\usebeamercolor{structure}\small
%       \draw[thick, fg, ->, >=latex] (0, 0) -- (0.8\textwidth, 0);
%       \draw[fg] (0, 0) node[anchor=north] {\footnotesize weaker};
%       \draw[fg] (0.8\textwidth, 0) node[anchor=north] {\footnotesize stronger};
      
%       \draw[fg, thick] (0.55\textwidth, -0.5ex) -- (0.55\textwidth, 0.5ex) node[above] {\bf envy-freeness};
%       \draw[fg, thick] (0.25\textwidth, -0.5ex) -- (0.25\textwidth, 0.5ex) node[above] {\bf proportional fair share};
%       \draw[fg, thick] (0.1\textwidth, -0.5ex) -- (0.1\textwidth, 0.5ex) node (MFS) {};
%       \draw[fg, <-, >=latex] (MFS) -- ++(0, 1.5em) node[above, draw=fg, fill=fg!30!bg, rounded corners] {\bf max-min fair share};
%     \end{tikzpicture}
%   \end{center}
% \end{frame}

\begin{frame}{Min-max fair share}
  \begin{itemize}
  \item Max-min fair share: \textit{``I cut, you choose (I choose last)''}\pause
  \item \textbf{Idea:} why not do the opposite (\textit{``Someone
      cuts, I choose first''}) ?\\
    \ \hfill $\to$ Min-max fair share
  \end{itemize}

  \pause\vfill

  \begin{block}{Min-max fair share (mFS)}
    The \emph{min-max fair share} of an agent $\ag$ is equal to:
    \[
    \umFS_\ag  \Def \min_{\Shares} \max_{\agp \in \Agents } u_\ag(\share_{\agp})
    \]
    
    An allocation $\Shares$ satisfies \emph{min-max fair share}
    (mFS) if every agent gets at least her min-max fair share.
  \end{block}    

  \pause\vfill

  \begin{itemize}
  \item mFS = the worst share an agent can get in a \textit{``Someone
      cuts, I choose first''} game.
  \item In the \emph{cake-cutting} case, same as PFS.
  \end{itemize}
\end{frame}

\begin{frame}{Min-max fair share: properties}
  \textbf{Facts:}
  \begin{itemize}
  \item Computing $\umFS_\ag$ for a given agent is hard $\to$
    \textbf{coNP}-complete [\textsc{Partition}]
  \item Hence, deciding whether an allocation satisfies mFS is also
    hard.
  \item $\Shares$ satisfies mFS $\Rightarrow$ $\Shares$ satisfies PFS.
  \item $\Shares$ is envy-free $\Rightarrow$ $\Shares$ satisfies mFS.
  \end{itemize}

  \pause\vfill
  
  \begin{center}
    \begin{tikzpicture}\usebeamercolor{structure}\small
      \draw[thick, fg, ->, >=latex] (0, 0) -- (0.8\textwidth, 0);
      \draw[fg] (0, 0) node[anchor=north] {\footnotesize weaker};
      \draw[fg] (0.8\textwidth, 0) node[anchor=north] {\footnotesize stronger};
      
      \draw[fg, thick] (0.55\textwidth, -0.5ex) -- (0.55\textwidth, 0.5ex) node[above] {\bf envy-freeness};
      \draw[fg, thick] (0.25\textwidth, -0.5ex) -- (0.25\textwidth, 0.5ex) node[above] {\bf proportional fair share};
      \draw[fg, thick] (0.1\textwidth, -0.5ex) -- (0.1\textwidth, 0.5ex) node (MFS) {};
      \draw[fg, <-, >=latex] (MFS) -- ++(0, 1.5em) node[above] {\bf max-min fair share};
      \draw[fg, thick] (0.4\textwidth, -0.5ex) -- (0.4\textwidth, 0.5ex) node (mFS) {};
      \draw[fg, <-, >=latex] (mFS) -- ++(0, 1.5em) node[above, draw=fg, fill=fg!30!bg, rounded corners] {\bf min-max fair share};
    \end{tikzpicture}
  \end{center}
\end{frame}

% \begin{frame}{Competitive Equilibrium from Equal Incomes}
%   \begin{bibunit}[apalike]
%     \begin{block}{Competitive Equilibrium from Equal Incomes (CEEI)}
%       \begin{itemize}
%       \item Set one price $p_\ob \leq \text{\EUR 1}$ for each object $\ob$.
%       \item Give \EUR 1 to each agent $\ag$.
%       \item Let $\share_\ag^\star$ be (among) the best share(s) agent
%         $\ag$ can buy with her \EUR 1.
%       \item If $(\share_1^\star, \dots, \share_\maxAgents^\star)$ is a
%         valid allocation, it forms, together with $\Prices$, a
%         \emph{CEEI}.
%       \end{itemize}

%       Allocation $\Shares$ satisfies CEEI if $\exists \Prices$ such
%       that $(\Shares, \Prices)$ is a CEEI.
%     \end{block}
    
%     \pause\vfill
    
%     \begin{itemize}
%     \item Classical notion in economics \cite{Moulin95}
%     \item Not so much studied in computer science (except \cite{Othman2010})
%     \end{itemize}
    
%     \vfill
    
%     \begin{beamercolorbox}[wd=\textwidth,sep=1ex]{custombox}%
%       \putbib[demo2bib]
%     \end{beamercolorbox}
%   \end{bibunit}
% \end{frame}

\begin{frame}{Competitive Equilibrium from Equal Incomes}
  \emph{Competitive Equilibrium from Equal Incomes}.\\
  \textbf{Example}: 4 objects $\{1, 2, 3, 4\}$, 2 agents $\{1, 2\}$.

  \vfill\pause
  
  \emph{Preferences:}

  \ 

  \begin{tabular}{|c|c|c|c|c|}
    \multicolumn{1}{c}{}
    & \multicolumn{1}{c}{\only<2>{\phantom{\tikz\node[rectangle callout, fill=structure.fg!80!bg, callout relative pointer={(0,-0.5)}] {\color{structure.fg!10!bg}\EUR 0.80};}}\only<3->{\tikz\node[rectangle callout, fill=structure.fg!80!bg, callout relative pointer={(0,-0.5)}] {\color{structure.fg!10!bg}\EUR 0.80};}}
    & \multicolumn{1}{c}{\only<2>{\phantom{\tikz\node[rectangle callout, fill=structure.fg!80!bg, callout relative pointer={(0,-0.5)}] {\color{structure.fg!10!bg}\EUR 0.20};}}\only<3->{\tikz\node[rectangle callout, fill=structure.fg!80!bg, callout relative pointer={(0,-0.5)}] {\color{structure.fg!10!bg}\EUR 0.20};}}
    & \multicolumn{1}{c}{\only<2>{\phantom{\tikz\node[rectangle callout, fill=structure.fg!80!bg, callout relative pointer={(0,-0.5)}] {\color{structure.fg!10!bg}\EUR 0.80};}}\only<3->{\tikz\node[rectangle callout, fill=structure.fg!80!bg, callout relative pointer={(0,-0.5)}] {\color{structure.fg!10!bg}\EUR 0.80};}}
    & \multicolumn{1}{c}{\only<2>{\phantom{\tikz\node[rectangle callout, fill=structure.fg!80!bg, callout relative pointer={(0,-0.5)}] {\color{structure.fg!10!bg}\EUR 0.20};}}\only<3->{\tikz\node[rectangle callout, fill=structure.fg!80!bg, callout relative pointer={(0,-0.5)}] {\color{structure.fg!10!bg}\EUR 0.20};}}
    \\    
    \hline
    & $1$ & $2$ & $3$ & $4$\\\hline
    agent $1$ & {7} & {\only<4->{\color{fg!30!bg}}2} & {\only<4->{\color{fg!30!bg}}6} & 10\\\hline
    agent $2$ & {\only<4->{\color{fg!30!bg}}7} & {6} & {8} & {\only<4->{\color{fg!30!bg}}4}\\\hline
  \end{tabular}

  \pause\vfill

  \textit{For \EUR 1, what would you buy?}

  \pause\vfill

  \parbox{0.4\textwidth}{
    \begin{itemize}
    \item\textbf{Agent 1: } \textbf{1} and \textbf{4};
    \item\textbf{Agent 2: } \textbf{2} and \textbf{3}.
    \end{itemize}
  }\pause\vrule\quad
  \parbox{0.4\textwidth}{
    \textbf{Disjoint shares!}
  }

  \pause\vfill

  $\Rightarrow$ Allocation $\langle \{1, 4\}, \{2, 3\} \rangle$, with
  prices $\langle 0.8, 0.2, 0.8, 0.2 \rangle$ forms a CEEI.

  $\Rightarrow$ Allocation $\langle \{1, 4\}, \{2, 3\} \rangle$
  satisfies CEEI.
\end{frame}

\begin{frame}{Competitive Equilibrium from Equal Incomes}
  \begin{bibunit}[apalike]
    \begin{itemize}
    \item Classical notion in economics \cite{Moulin95}
    \item Not so much studied in computer science (except \cite{Othman2010})
    \end{itemize}
    
    \vfill
    
    \begin{beamercolorbox}[wd=\textwidth,sep=1ex]{custombox}%
      \putbib[demo2bib]
    \end{beamercolorbox}
  \end{bibunit}  

  \vfill

  Complexity supposedly hard, but still \textbf{open} (?).

  \vfill%\pause

  \textbf{Fact:} $\Shares$ satisfies CEEI $\Rightarrow$ $\Shares$ is envy-free.

  \vfill\pause

  \begin{center}
    \begin{tikzpicture}\usebeamercolor{structure}\small
      \draw[thick, fg, ->, >=latex] (0, 0) -- (0.8\textwidth, 0);
      \draw[fg] (0, 0) node[anchor=north] {\footnotesize weaker};
      \draw[fg] (0.8\textwidth, 0) node[anchor=north] {\footnotesize stronger};
      
      \draw[fg, thick] (0.55\textwidth, -0.5ex) -- (0.55\textwidth, 0.5ex) node[above] {\bf envy-freeness};
      \draw[fg, thick] (0.25\textwidth, -0.5ex) -- (0.25\textwidth, 0.5ex) node[above] {\bf proportional fair share};
      \draw[fg, thick] (0.1\textwidth, -0.5ex) -- (0.1\textwidth, 0.5ex) node (MFS) {};
      \draw[fg, <-, >=latex] (MFS) -- ++(0, 1.5em) node[above] {\bf max-min fair share};
      \draw[fg, thick] (0.4\textwidth, -0.5ex) -- (0.4\textwidth, 0.5ex) node (mFS) {};
      \draw[fg, <-, >=latex] (mFS) -- ++(0, 1.5em) node[above] {\bf min-max fair share};
      \draw[fg, thick] (0.7\textwidth, -0.5ex) -- (0.7\textwidth, 0.5ex) node (CEEI) {};
      \draw[fg, <-, >=latex] (CEEI) -- ++(0, 1.5em) node[above, draw=fg, fill=fg!30!bg, rounded corners] {\bf CEEI};
    \end{tikzpicture}
  \end{center}
\end{frame}

\begin{frame}{Interpretation}
  \begin{center}
    \begin{tikzpicture}\usebeamercolor{structure}\small
      \draw[thick, fg, ->, >=latex] (0, 0) -- (0.8\textwidth, 0);
      \draw[fg] (0, 0) node[anchor=north] {\footnotesize weaker};
      \draw[fg] (0.8\textwidth, 0) node[anchor=north] {\footnotesize stronger};
      
      \draw[fg, thick] (0.55\textwidth, -0.5ex) -- (0.55\textwidth, 0.5ex) node[above] {\bf envy-freeness};
      \draw[fg, thick] (0.25\textwidth, -0.5ex) -- (0.25\textwidth, 0.5ex) node[above] {\bf proportional fair share};
      \draw[fg, thick] (0.1\textwidth, -0.5ex) -- (0.1\textwidth, 0.5ex) node (MFS) {};
      \draw[fg, <-, >=latex] (MFS) -- ++(0, 1.5em) node[above] {\bf max-min fair share};
      \draw[fg, thick] (0.4\textwidth, -0.5ex) -- (0.4\textwidth, 0.5ex) node (mFS) {};
      \draw[fg, <-, >=latex] (mFS) -- ++(0, 1.5em) node[above] {\bf min-max fair share};
      \draw[fg, thick] (0.7\textwidth, -0.5ex) -- (0.7\textwidth, 0.5ex) node (CEEI) {};
      \draw[fg, <-, >=latex] (CEEI) -- ++(0, 1.5em) node[above] {\bf CEEI};
    \end{tikzpicture}
  \end{center}

  \vfill

  \begin{enumerate}
  \item<2-> For all allocation $\Shares$:
    \[
    (\Verif{\CEEI}{\Shares}) \Rightarrow (\Verif{\EF}{\Shares})
    \Rightarrow (\Verif{\mFS}{\Shares}) \Rightarrow
    (\Verif{\PFS}{\Shares}) \Rightarrow (\Verif{\MFS}{\Shares})
    \]
    \ \hfill$\to$ the highest property $\Shares$ satisfies, the most
    satisfactory it is.
  \item<3-> If $\Inst{\mathcal{P}}$ is the set of instances s.t at
    least one allocation satisfies $\mathcal{P}$:
    \[
    \InstCEEI \subset \InstEF \subset
    \InstmFS \subset \InstPFS \subset \InstMFS \subset \Instances
    \]
    \ \hfill$\to$ the lowest subset, the less ``conflict-prone''.
  \end{enumerate}

  \vfill

  \begin{overlayarea}{\textwidth}{3cm}
    \only<4->{
      \textbf{Two extreme examples:}
      \begin{itemize}
      \item 2 agents, 1 object $\to$ only in $\InstMFS$
      \item 2 agents, 2 objects, with\\
        \begin{tabular}{|c|c|c|}
          \hline
          & $1$ & $2$\\\hline
          agent $1$ & 1000 & 0\\\hline
          agent $2$ & 0 & 1000\\\hline
        \end{tabular}
        $\to$ in $\InstCEEI$ (with \textit{e.g.} $\Prices = \langle 1,
        1\rangle$).
      \end{itemize}
    }
  \end{overlayarea}
\end{frame}


% \section{Additional properties}

% % \begin{frame}{Outline}
% %   \tableofcontents[current]
% % \end{frame}


% \begin{frame}{Additional properties}
%   \emph{1. Strict inclusions}\hfill
%   $\InstCEEI \subset \InstEF \subset
%   \InstmFS \subset \InstPFS \subset \InstMFS (= \Instances ?)$\\

%   Are these inclusions strict? \pause Yes, they are, and we can prove
%   it!

%   \pause\vfill
  
%   \emph{2. Properties and egalitarianism?}
  
%   \begin{itemize}
%   \item\emph{Envy-freeness:} question studied in [Brams and King, 2005]
%   \item\emph{Max-min fair share:} egalitarian optimal
%     allocations \emph{almost always satisfy} max-min fair share.
%   \end{itemize}
  
%   \pause\vfill
  
%   \emph{3. Interpersonal comparison}

%   \begin{itemize}
%   \item Egalitarianism requires the preferences to be
%     \emph{comparable}:
%     \begin{itemize}
%     \item either expressed on a same scale (\textit{e.g.} money)...
%     \item ...or normalized (\textit{e.g.} Kalai-Smorodinsky)
%     \end{itemize}
%   \item The five fairness criteria introduced do not
%     (\emph{independence of the individual utility scales}).
%   \end{itemize}

%   $\to$ This is a very appealing property.

% \end{frame}


% \begin{frame}{Strict inclusions?}  
%   \[
%   \InstCEEI \subset \InstEF \subset
%   \InstmFS \subset \InstPFS \subset \InstMFS (= \Instances ?)
%   \]

%   \ 

%   Are these inclusions strict?

%   \pause

%   \begin{itemize}
%   \item From MFS to PFS: two agents, one object.
%   \item From PFS to mFS: an example with 3 agents, 3 objects
%     found.
%   \item From mFS to EF: not straightforward, but one example with
%     3 agents, 4 objects found.
%   \item From EF to CEEI: no example found\fup{1}, but very likely to
%     be strict by computational complexity arguments.
%   \end{itemize}

%   \vfill

%   \hrule
%   \vskip0.1cm
%   {\footnotesize\fup{1} because it seems algorithmically hard to
%     compute a CEEI...}
% \end{frame}

% \begin{frame}{Back to egalitarianism...}
%   Other approach to fairness... Find an allocation $\Shares$ that:
%   \begin{enumerate}
%   \item maximizes the collective utility defined by a \emph{collective utility function},\\
%     \ \hfill \textit{e.g.} $uc(\Shares) = \min_{\ag \in \Agents}
%     u(\share_\ag)$ -- egalitarian solution
%   \end{enumerate}
  
%   \pause\vfill

%   To which extent is it compatible with the property-based approach?

%   \pause\vfill
  
%   \begin{bibunit}[apalike]
%     \begin{itemize}
%     \item<3-> \emph{Envy-freeness:} question studied in \cite{BramsKing05}
%     \item<4-> \emph{Max-min fair share:} egalitarian optimal
%       allocations \emph{almost always satisfy} max-min fair share.\\
%       \ \\
%       \begin{tabular}{|c|c|c|c|c|l}
%         \cline{1-5}
%         & 1 & 2 & 3 & 4 & \\\cline{1-5}
%         agent 1 & 58 &\dag 15 &\dag *19 &  8  & $\rightarrow$ *19 / \dag34\\\cline{1-5}
%         agent 2 & \dag 63 &*5 & 25 & *7 & $\rightarrow$ *12 / \dag63 \\\cline{1-5}
%         agent 3 & 37 &10 & *27 & \dag26 & $\rightarrow$ *27 / \dag26 \\\cline{1-5}
%       \end{tabular}\\
%       {
%         \footnotesize
%         \textbf{3 agents, 4 objects:}
%         about 1 counterexample for 3500 instances
%       }
%     \end{itemize}
    
%     \vfill
    
%     \begin{beamercolorbox}[wd=\textwidth,sep=1ex]{custombox}%
%       \putbib[demo2bib]
%     \end{beamercolorbox}
%   \end{bibunit}
% \end{frame}

% \begin{frame}{Interpersonal comparison}
%   \textbf{Note:}
%   \begin{itemize}
%   \item Egalitarianism requires the preferences to be
%     \emph{comparable}:
%     \begin{itemize}
%     \item either expressed on a same scale (\textit{e.g.} money)...
%     \item ...or normalized (\textit{e.g.} Kalai-Smorodinsky)
%     \end{itemize}
%   \item The five fairness criteria introduced do not
%     (\emph{independence of the individual utility scales}).
%   \end{itemize}

%   $\to$ This is a very appealing property.
% \end{frame}


% \section{A glimpse beyond additive preferences}

% \begin{frame}{Outline}
%   \tableofcontents[current]
% \end{frame}

% % \begin{frame}{$k$-additive preferences}
% %   \begin{itemize}
% %   \item Additive preferences are nice but have a limited
% %     expressiveness.
% %   \item<2-> \textbf{Examples:}
% %     \begin{itemize}
% %     \item the pair of skis and the pair of ski poles (complementarity)\\
% %       \ \hfill\ \only<3->{$\to$ $u(\{skis, poles\}) > u(skis) + u(poles)$}
% %     \item the pair of skis and the snowboard (substitutability)\\
% %       \ \hfill\ \only<4->{$\to$ $u(\{skis, snowboard\}) < u(skis) +
% %         u(snowboard)$}
% %     \end{itemize}
% %   \end{itemize}

% %   \vfill

% %   \begin{overlayarea}{\textwidth}{2cm}
% %     \only<5->{
% %       \begin{block}{$k$-additive preferences}
% %         A weight $w(\mathcal{S})$ to each subset $\mathcal{S}$ of
% %         objects (not only singletons) of size $\leq k$.

% %         \textbf{Note:} additive = 1-additive
% %       \end{block}
% %     }
% %   \end{overlayarea}
  
% %   \vfill

% %   \begin{overlayarea}{\textwidth}{3cm}
% %     \only<6->{
% %       \textbf{Examples:}
% %       \begin{itemize}
% %       \item $w(skis) = 10$; $w(poles) = 0$; $w(\{skis, poles\}) = 90$\\
% %         \ \hfill $\to$ $u(\{skis, poles\}) = 100 > 10 + 0$
% %       \item $w(skis) = 100$; $w(snowboard) = 100$; $w(\{skis, snowboard\})
% %         = -100$\\
% %         \ \hfill $\to$ $u(\{skis, snowboard\}) = 100 < 100 + 100$
% %       \end{itemize}
% %     }
% %   \end{overlayarea}
% % \end{frame}

% \begin{frame}{MFS and $k$-additive preferences}
%   \begin{block}{Reminder}
%     For \textbf{additive preferences} we can almost always find an allocation
%     satisfying max-min fair share.
%   \end{block}
  
%   \pause\vfill

%   For $k$-additive preferences ($k \geq 2$) this is obviously not true:\\

%   \ 

%   \textbf{Example:} 4 objects, 2 agents\\
%   \scalebox{0.8}{
%     \begin{tikzpicture}
%       \useasboundingbox (-0.25, -0.75) rectangle (5, 2.75);
%       \draw[thick] (0, 0) node[draw, cross out, inner sep=1pt] (a) {};
%       \draw[thick] (2, 0) node[draw, cross out, inner sep=1pt] (b) {};
%       \draw[thick] (2, 2) node[draw, cross out, inner sep=1pt] (c) {};
%       \draw[thick] (0, 2) node[draw, cross out, inner sep=1pt] (d) {};
      
%       \draw (a.north) node[anchor=south] {1};
%       \draw (b.north) node[anchor=south] {2};
%       \draw (c.north) node[anchor=south] {3};
%       \draw (d.north) node[anchor=south] {4};
      
%       \only<3->{
%         \draw[opacity=0.4] (1, 0) node[draw=blue, thick, fill=blue!50!white, shape=ellipse, minimum width=2.5cm, minimum height=1cm] {};
%         \draw[opacity=0.4] (1, 2) node[draw=blue, thick, fill=blue!50!white, shape=ellipse, minimum width=2.5cm, minimum height=1cm] {};
%         \draw[blue] (3, 2) node[anchor=west] {Agent 1: $w(\{1,2\}) = w(\{3,4\}) = 1$ $\to$ $\uMFS_1 = 1$};
%       }
      
%       \only<4->{
%         \draw[opacity=0.4] (0, 1) node[draw=red, thick, fill=red!50!white, shape=ellipse, minimum height=2.5cm, minimum width=1cm] {};
%         \draw[opacity=0.4] (2, 1) node[draw=red, thick, fill=red!50!white, shape=ellipse, minimum height=2.5cm, minimum width=1cm] {};
%         \draw[red] (3, 1.5) node[anchor=west] {Agent 2: $w(\{1,4\}) = w(\{2,3\}) = 1$  $\to$ $\uMFS_2 = 1$};
%       }
%     \end{tikzpicture}
%   }

%   \begin{overlayarea}{\textwidth}{1cm}
%     \only<5->{
%       Worse\dots Deciding whether there exists one is
%       \textbf{NP}-complete [\textsc{Partition}].
%     }
%   \end{overlayarea}
% \end{frame}

\section{Conclusion}

% \begin{frame}{Outline}
%   \tableofcontents[current]
% \end{frame}

\begin{frame}{Take-away message}
  A scale of properties (for numerical additive preferences)...

  \scalebox{0.7}{
    \begin{tikzpicture}
      \useasboundingbox (0, 0) rectangle (10, 5);
      \only<2->{\draw[opacity=0.4, thick,draw=red, fill=red!50!white] (0, 0) rectangle (1.5, 1);}
      \only<3->{\draw[opacity=0.4, thick,draw=red!50!orange, fill=red!50!orange!50!white] (0, 1) rectangle (1.5, 2);}
      \only<4->{\draw[opacity=0.4, thick,draw=orange, fill=orange!50!white] (0, 2) rectangle (1.5, 3);}
      \only<5->{\draw[opacity=0.4, thick,draw=orange!50!green, fill=orange!50!green!50!white] (0, 3) rectangle (1.5, 4);}
      \only<6->{\draw[opacity=0.4, thick,draw=green!60!black, fill=green!60!black!50!white] (0, 4) rectangle (1.5, 5);}
      
      \only<2->{\draw (2, 0.5) node[anchor=west, text width=8cm] {{\color{red} \bf Max-min fair share}\\Almost always possible to satisfy it};}
      \only<3->{\draw (2, 1.5) node[anchor=west, text width=8cm] {{\color{red!50!orange}\bf Proportional fair share}\\Cannot be satisfied \textit{e.g.} in the 1 object, 2 agents case};}
      \only<4->{\draw[orange] (2, 2.5) node[anchor=west] {\bf Min-max fair share};}
      \only<5->{\draw (2, 3.5) node[anchor=west, text width=8cm] {{\color{orange!50!green}\bf Envy-freeness}\\Requires somewhat complementary preferences};}
      \only<6->{\draw (2, 4.5) node[anchor=west, text width=8cm] {{\color{green!60!black}\bf Competitive Equilibrium from Equal Incomes}\\Requires complementary preferences};}
    \end{tikzpicture}
  }

  \vfill

  \begin{overlayarea}{\textwidth}{2.5cm}
    \only<7->{
      A possible approach to fairness in multiagent resource allocation problems:
      \begin{enumerate}
      \item Determine the highest satisfiable criterion.
      \item Find an allocation that satisfies this criterion.
      \item Explain to the upset agents that we cannot do much better.
      \end{enumerate}
    }
  \end{overlayarea}
\end{frame}

\begin{frame}{What else}
  Some other results (see the paper)\dots
  \begin{itemize}
  \item All the inclusions are strict
  \item Link with egalitarianism
  \item Experimental results
  \item A glimpse beyond additive preferences
  \end{itemize}

  \pause\vfill

  Some future directions\dots

  \begin{itemize}
  \item Some missing complexity results.
  \item Develop efficient \emph{algorithms}
  \item More \emph{experiments}
  \item Extend to \emph{more expressive preference languages}
    (including ordinal ones...).
  \end{itemize}
\end{frame}

\end{document}

%%% Local Variables: 
%%% mode: latex
%%% TeX-master: t
%%% End: 
