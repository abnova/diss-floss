\usepackage{stmaryrd}                  % Pour avoir les d\'elimiteurs entiers
\usepackage{booktabs}
\usepackage{tabularx}
\usepackage{xspace}

\newcommand{\Omit}[1]{}
\newcommand{\fixme}[1]{{\small\color{red}[FIXME : #1]}}
\definecolor{mybeautifulcyan}{rgb}{0.08,0.58,0.89}
\newcommand{\sylvain}[1]{{\small\color{mybeautifulcyan}[Sylvain : #1]}}
\newcommand{\michel}[1]{{\small\color{green}[Michel : #1]}}

% --------- ensembles célèbres
\newcommand{\N}{\mathbb{N}}
\newcommand{\R}{\mathbb{R}}
\newcommand{\Rplus}{\R^+}


% -------- variables
%\newcommand{\Var}[1]{\mathbf{#1}}
%\newcommand{\Dom}[1]{\mathrm{Dom}(#1)}
%\newcommand{\EnsVar}[1]{\mathbf{#1}}

\newcommand{\ens}[1]{\mathcal{#1}}
%\newcommand{\vect}[1]{\boldsymbol{\mathbf{#1}}}        
\newcommand{\vect}[1]{\overrightarrow{#1}}
\newcommand{\setsof}[1]{\wp(#1)}     
\newcommand{\mat}[1]{\vect{#1}}
%\newcommand{\matrices}[3]{\mathcal{M}_{#1, #2}(#3)}
\newcommand{\multiens}[1]{\{\!\!\{#1\}\!\!\}}

%--------- logique
% \newcommand{\If}{\mathbf{if}}
% \newcommand{\Otherwise}{\mathbf{otherwise}}

%--------- divers
\newcommand{\Def}{\buildrel\hbox{\tiny \textit{def}}\over =}
\newcommand{\tq}{\ |\ }
\DeclareMathOperator*{\esp}{E}
\DeclareMathOperator*{\esppr}{E_{\Pr}}
\DeclareMathOperator{\argmax}{argmax}
\DeclareMathOperator{\argmin}{argmin}
\DeclareMathOperator{\leximin}{leximin}
\DeclareMathOperator{\sign}{sign}
\DeclareMathOperator{\variance}{Var}
\newcommand{\Variance}[1]{\variance(#1)}

\newtheorem{defin}{Définition}
\newtheorem{prop}{Proposition}
\newtheorem{exemple}{Exemple}
\newtheorem{remarque}{Remarque}
\newtheorem{lemme}{Lemme}
\newtheorem{conjecture}{Conjecture}
\newtheorem{corollaire}{Corollaire}

%\renewcommand{\qedsymbol}{$\blacksquare$}
\theoremstyle{nonumberplain}
%\theorembodyfont{\normalfont}
%\theoremsymbol{\qedsymbol}

%------------ macros specifiques pour l'article

% lettre pour agents, objets
\newcommand{\Ag}{A} 
\newcommand{\Ob}{O}
\newcommand{\ag}{i}
\newcommand{\agp}{j}
\newcommand{\ob}{o}

\newcommand{\Agents}{{\ens{\Ag}}}
\newcommand{\maxAgents}{{n}}
\newcommand{\Objets}{{\ens{\Ob}}}
\newcommand{\maxObjets}{{m}}

\newcommand{\score}{g}
\newcommand{\rang}{r}

\newcommand{\Poids}{W}
\newcommand{\poids}{w}

\newcommand{\share}{\pi}
\newcommand{\Shares}{\vect{\share}}

\newcommand{\price}{p}
\newcommand{\Prices}{\vect{\price}}

\newcommand{\Ut}{u}     % utilité individuelle
\newcommand{\Utc}{cu}   % utilité collective

\newcommand{\pb}{problème de partage de biens indivisibles à
préférences additives\xspace}
\newcommand{\jpp}{juste part proportionnelle\xspace}
\newcommand{\jpM}{juste part max-min\xspace}
\newcommand{\jpm}{juste part min-max\xspace}


\newcommand{\PFS}{\mathrm{PFS}}   % proportional fair share property
\newcommand{\MFS}{\mathrm{MFS}} % maxmin fair share property
\newcommand{\mFS}{\mathrm{mFS}}  % minmax fair share property
\newcommand{\EF}{\mathrm{EF}}  % envy-free property
\newcommand{\CEEI}{\mathrm{CEEI}}  % CEEI property

\newcommand{\Instances}{\mathcal{I}}
\newcommand{\Inst}[1]{\mathcal{I}_{| #1}}
\newcommand{\InstPFS}{\Inst{\PFS}}   % proportional fair share property
\newcommand{\InstMFS}{\Inst{\MFS}} % maxmin fair share property
\newcommand{\InstmFS}{\Inst{\mFS}}  % minmax fair share property
\newcommand{\InstEF}{\Inst{\EF}}  % envy-free property
\newcommand{\InstCEEI}{\Inst{\CEEI}}  % CEEI property

\newcommand{\Verif}[2]{#2 \vDash #1}

\newcommand{\uPFS}{u^\PFS}   
\newcommand{\uMFS}{u^\MFS}
\newcommand{\umFS}{u^\mFS}

\newcommand{\Adm}{{\cal F}}

\newcommand{\llb}{\llbracket}
\newcommand{\rrb}{\rrbracket}

\newcommand{\pol}{\textsf{P}\xspace}
\newcommand{\np}{\textsf{NP}\xspace}
\newcommand{\conp}{\textsf{coNP}\xspace}
\newcommand{\DP}{\textsf{DP}\xspace}
\newcommand{\bhdeux}{$\mathsf{BH_2}$\xspace}
\newcommand{\Prob}[1]{\textsc{[#1]}}
\newcommand{\sigmaDeuxP}{$\mathsf{\Sigma_2^P}$\xspace}

\newcolumntype{L}{>{\raggedright\arraybackslash}X}
\newcolumntype{R}{>{\raggedleft\arraybackslash}X}
\newcolumntype{C}{>{\centering\arraybackslash}X}

\newcounter{problem}
\newcommand{\EP}[3]{
  \refstepcounter{problem}
  \smallskip
  \begin{center}
    {\small
      \begin{tabularx}{\columnwidth}{ll}
        \toprule
        \multicolumn{2}{c}{Problème~\theproblem\quad\Prob{#1}} \\
        \midrule
        {\bf Entrée :} & \parbox[t]{0.70\columnwidth}{#2\vspace*{1mm}} \\
        {\bf Question :}& \parbox[t]{0.70\columnwidth}{#3\vspace*{.5mm}} \\
        \bottomrule
      \end{tabularx}
    }
  \end{center}
  \smallskip
}

